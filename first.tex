\documentclass[11pt,a4paper]{article}
\usepackage[utf8]{inputenc}[spanish]
\usepackage{amsmath}
\usepackage{amsfonts}
\usepackage{amssymb}
\usepackage{lmodern}
\usepackage{amsmath}
\usepackage{amsthm}
\usepackage{enumerate}
\usepackage{graphicx}
\usepackage{mathtools}
\usepackage{stackrel}
\usepackage{multicol}
\usepackage[left=1cm,right=1cm,top=1cm,bottom=1cm]{geometry}
\newcounter{neq}
\providecommand{\abs}[1]{\lvert#1\rvert}
\newcommand{\SIGMA}{\Sigma^{\ast}}
\newcommand{\PN}{\par\noindent}

\title{Resumen Física}

\begin{document}
	\begin{center}
		\Huge \textbf{Resumen Física: Primer Parcial} \\
		\large Agustín Curto
	\end{center}

	\section*{\underline{Práctico 0:} Vectores}
		\PN Sean $\vec{A} = A_{x} \hat{i} + A_{y} \hat{j} + A_{z} \hat{k}$, $\;$ $\vec{B} = B_{x} \hat{i} + B_{y} \hat{j} +
			B_{z} \hat{k}$, $\;$ $\theta = \vec{A} \angle \vec{B}$, $\;$ $A$ y $B$ los módulos de $\vec{A}, \vec{B}$
			respectivamente y $\alpha$ un escalar.
		\begin{multicols}{2}
			\begin{itemize}
				\item Suma: $\vec{A} + \vec{B} = (A_{x} + B_{x}) \hat{i} + (A_{y} + B_{y}) \hat{j} + (A_{z} + B_{z}) \hat{k}$
				\item Resta: $\vec{A} - \vec{B} = (A_{x} - B_{x}) \hat{i} + (A_{y} - B_{y}) \hat{j} + (A_{z} - B_{z}) \hat{k}$
				\item Multiplicación por un escalar: $\alpha A_{x} \hat{i} + \alpha A_{y} \hat{j} + \alpha A_{z} \hat{k}$
				\item Producto escalar (punto): $\vec{A} \cdot \vec{B} = AB \cos(\theta)$
				\item Producto vectorial (cruz):
					\PN \underline{Forma 1:}
					\[
						\vec{A} \times \vec{B} = (A_{y}B_{z} - A_{z}B_{y}) \hat{i} + (A_{x}B_{z} - A_{z}B_{x}) \hat{j} +
						(A_{x}B_{y} - A_{y}B_{x}) \hat{k}
					\]

					\PN \underline{Forma 2:}
						\[
							\vec{A} \times \vec{B} = det
								\left(
								  \begin{bmatrix}
								    \hat{i} & \hat{j} & \hat{k} \\
								    A_{x} & A_{y} & A_{z} \\
								    B_{x} & B_{y} & B_{z}
								  \end{bmatrix}
								\right)
						\]
					\PN Además: $\abs{\vec{A} \times \vec{B}} = AB \sin(\theta)$
				\item Representación polar:
					\begin{eqnarray*}
						a_{x} = r \cos(\theta) \\
						a_{y} = r \sin(\theta) \\
						r = \sqrt(a_{x}^{2} + a_{y}^{2})
					\end{eqnarray*}
			\end{itemize}
		\end{multicols}

	\section*{\underline{Práctico 1:} Cinemática}
		\begin{multicols}{2}
			\begin{itemize}
				\item Velocidad media: $\overline{v} = \frac{\Delta x}{\Delta t}$
				\item Aceleración media: $\overline{a} = \frac{\Delta v}{\Delta t}$
			\end{itemize}
		\end{multicols}

		\subsubsection*{Ecuaciones de movimiento:}
			\begin{multicols}{3}
				\begin{itemize}
					\item \underline{Posición:} $x(t) = x_{0} + v_{0} t + \frac{a}{2} t^{2}$
					\item \underline{Velocidad:} $v(t) = v_{0} + at$
					\item \underline{Aceleración:} $a = cte$
				\end{itemize}
			\end{multicols}

	\section*{Práctico 2: Movimiento en el plano}
		\subsubsection*{Ecuaciones de movimiento:}
			\begin{itemize}
				\item Posición: $\vec{r}(t) = x(t) + y(t)$
				\item Velocidad: $\vec{v}(t) = v_{x}(t) \hat{i} + v_{y}(t) \hat{j}$
				\item Aceleración: $\vec{a}(t) = a_{x}(t) \hat{i} + a_{y}(t) \hat{j}$
			\end{itemize}

	\section*{Práctico 3: Movimiento circular}
		\begin{itemize}
			\item
		\end{itemize}

	\section*{Práctico 4: Dinámica}
		\begin{itemize}
			\item
		\end{itemize}
\end{document}
