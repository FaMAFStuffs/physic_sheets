\documentclass[12pt,a4paper]{article}
\usepackage[utf8]{inputenc}[spanish]
\usepackage{amsmath}
\usepackage{amsfonts}
\usepackage{amssymb}
\usepackage{lmodern}
\usepackage{amsmath}
\usepackage{amsthm}
\usepackage{enumerate}
\usepackage{graphicx}
\usepackage{mathtools}
\usepackage{stackrel}
\usepackage{multicol}
\usepackage[left=1cm,right=1cm,top=1cm,bottom=1cm]{geometry}
\newcounter{neq}
\providecommand{\abs}[1]{\lvert#1\rvert}
\newcommand{\SIGMA}{\Sigma^{\ast}}
\newcommand{\PN}{\par\noindent}

\title{Resumen Física}

\begin{document}
	\begin{center}
		\Huge \textbf{Resumen Física: Segundo Parcial} \\
		\large Agustín Curto
	\end{center}

	\section*{\underline{Práctico 0:} Vectores}
		\PN $\vec{A} = A_{x} \hat{i} + A_{y} \hat{j} + A_{z} \hat{k}$, $\;$ $\vec{B} = B_{x} \hat{i} + B_{y} \hat{j} +
			B_{z} \hat{k}$, $\;$ $\theta = \vec{A} \angle \vec{B}$, $\;$ $A = \abs{\vec{A}}, B = \abs{\vec{B}}$, $\alpha$.
		\begin{multicols}{2}
			\begin{itemize}
				\item Suma: $\vec{A} + \vec{B} = (A_{x} + B_{x}) \hat{i} + (A_{y} + B_{y}) \hat{j} + (A_{z} + B_{z}) \hat{k}$
				\item Resta: $\vec{A} - \vec{B} = (A_{x} - B_{x}) \hat{i} + (A_{y} - B_{y}) \hat{j} + (A_{z} - B_{z}) \hat{k}$
				\item Multiplicación por un escalar: $\alpha A_{x} \hat{i} + \alpha A_{y} \hat{j} + \alpha A_{z} \hat{k}$
				\item Producto escalar (punto): $\vec{A} \cdot \vec{B} = AB \cos(\theta)$
				\item Producto vectorial (cruz):
					\PN \underline{Forma 1:}
					\[
						\vec{A} \times \vec{B} = (A_{y}B_{z} - A_{z}B_{y}) \hat{i} + (A_{x}B_{z} - A_{z}B_{x}) \hat{j} +
						(A_{x}B_{y} - A_{y}B_{x}) \hat{k}
					\]

					\PN \underline{Forma 2:}
						\[
							\vec{A} \times \vec{B} = det
								\left(
								  \begin{bmatrix}
								    \hat{i} & \hat{j} & \hat{k} \\
								    A_{x} & A_{y} & A_{z} \\
								    B_{x} & B_{y} & B_{z}
								  \end{bmatrix}
								\right)
						\]
					\PN Además: $\abs{\vec{A} \times \vec{B}} = AB \sin(\theta)$
				\item Representación polar:
					\begin{eqnarray*}
						a_{x} = r \cos(\theta) \\
						a_{y} = r \sin(\theta) \\
						r = \sqrt(a_{x}^{2} + a_{y}^{2})
					\end{eqnarray*}
			\end{itemize}
		\end{multicols}

	\section*{\underline{Práctico 1:} Cinemática}
		\begin{multicols}{2}
			\begin{itemize}
				\item Velocidad media: $\overline{v} = \frac{\Delta x}{\Delta t}$
				\item Aceleración media: $\overline{a} = \frac{\Delta v}{\Delta t}$
			\end{itemize}
		\end{multicols}

		\subsubsection*{Ecuaciones de movimiento:}
			\begin{multicols}{3}
				\begin{itemize}
					\item \underline{Posición:} $x(t) = x_{0} + v_{0} t + \frac{a}{2} t^{2}$
					\item \underline{Velocidad:} $v(t) = v_{0} + at$
					\item \underline{Aceleración:} $a = cte$
				\end{itemize}
			\end{multicols}

	\section*{\underline{Práctico 2:} Movimiento en el plano}
		\begin{itemize}
			\item Trayectoria: $y(x)$.
		\end{itemize}

		\subsubsection*{Ecuaciones de movimiento:}
			\begin{itemize}
				\item \underline{Posición:} $\vec{r}(t) = x(t) \hat{i} + y(t) \hat{j}$
				\item \underline{Velocidad:} $\vec{v}(t) = v_{x}(t) \hat{i} + v_{y}(t) \hat{j}$
				\item \underline{Aceleración:} $\vec{a}(t) = a_{x}(t) \hat{i} + a_{y}(t) \hat{j}$
			\end{itemize}

	\section*{\underline{Práctico 3:} Movimiento circular}
		\PN $v = \abs{\vec{v}}$
		\begin{multicols}{2}
			\begin{itemize}
				\item Aceleración: $\vec{a} = \vec{a_{c}} + \vec{a_{t}} \;$ donde:
					\begin{itemize}
						\item $\vec{a_{c}} = \frac{v^{2}}{r}$
						\item $\abs{\vec{a_{c}}} = r\gamma = r \frac{d\omega}{dt} = r \ddot{\theta}$
						\item $\vec{a_{t}} = \frac{d\vec{v_{t}}}{dt}$
						\item $\abs{\vec{a_{t}}} = r\omega^{2} = r \dot{\theta^{2}}$
					\end{itemize}
				\item Velocidad angular: $\omega = \frac{v}{r} \; [\frac{rad}{sec}]$
				\item Velocidad tangencial: $v_{t} = \omega r \; [\frac{mts}{sec}]$
				\item Período: $T = \frac{2\pi}{\omega}$
				\item Frecuencia: $f = \frac{1}{T}$
				\item Perímetro: $P = 2\pi r$
			\end{itemize}
		\end{multicols}

		\subsubsection*{Ecuaciones de movimiento en coordenadas polares:}
			\PN $\hat{r} = \neg\hat{n}$, $\hat{\theta} = \hat{t}$. $\dot{r} = \frac{dr}{dt}$, $\dot{\theta} =
				\frac{d\theta}{dt} = \omega$.
			\begin{itemize}
				\item \underline{Posición:} $\vec{r}(t) = r(t) \hat{r}$
				\item \underline{Velocidad:} $\vec{v}(t) = \dot{r}(t) \hat{r} + r(t) \dot{\theta}(t) \hat{\theta}$
				\item \underline{Aceleración:} $\vec{a}(t) = (\ddot{r}(t) - r(t) \dot{\theta}^{2}(t)) \hat{r} + (r(t)
					\ddot{\theta}(t) + 2 \dot{r}(t) \dot{\theta}(t)) \hat{\theta}$
			\end{itemize}

	\section*{\underline{Práctico 4:} Dinámica}
		\begin{multicols}{2}
			\begin{itemize}
				\item Leyes de Newton:
				\begin{enumerate}
					\item $\sum_{i} \vec{F_{i}} = 0 \Rightarrow \vec{a} = 0$ y $\vec{v} = cte$
					\item $\vec{F} = m \vec{a} \Rightarrow \sum \vec{F} = m \vec{a}$
				\end{enumerate}
				\item Fuerza gravitacional: $\vec{P} = m \vec{g}$
				\item Fuerza de rozamiento: $\abs{\vec{F_{R}}} = \mu \abs{\vec{N}}$
				\item Fuerza centrípeta: $\vec{F_{c}} = m \vec{a_{c}}$
				\item Resortes: $F = k_{e} \Delta x$
				\begin{itemize}
					\item En paralelo: $k_{e} = \sum_{i} k_{i}$
					\item En serie: $k_{e} = \frac{1}{\sum_{i} \frac{1}{k_{e}}}$
				\end{itemize}
			\end{itemize}
		\end{multicols}

	\section*{\underline{Práctico 5:} Trabajo y Energía}
		\begin{multicols}{2}
			\begin{itemize}
				\item Trabajo: $W = \int_{pos_{i}}^{pos_{f}} \vec{F} d\vec{s}$
				\item Trabajo: $W = F \Delta x$, $\; [W] = \frac{N}{m} = Joule$
				\item Energía Cinética: $K = \frac{1}{2} m v^{2}$, $\; [K] = Joule$
				\item Energía Potencial: $U = mgh$, $\; [U] = Joule$
				\item Teorema de Energía-Trabajo:
					\[
						F \Delta x = \frac{1}{2} m v_{f}^{2} - \frac{1}{2} m v_{i}^{2}
					\]
				\item Conservación de la Energía: $E_{inicial} = E_{final}$, de otra forma $K_{i} + U_{i} = K_{f} + U_{f}$
				\item Potencia: $\vec{P} = \vec{F} \cdot \vec{v}$, $\; [P] = \frac{J}{s} = Watt$
				\item Potencia media: $\overline{P} = \frac{W}{\Delta t}$
				\item Potencia instantanea: $P = \frac{dT}{dt}$
			\end{itemize}
		\end{multicols}

	\section*{\underline{Práctico 6:} Momento Lineal y Angular}
		\begin{multicols}{2}
			\begin{itemize}
				\item Momento lineal: $\vec{p} = m \vec{v}$, $\; [\vec{p}] = [Kg] [\frac{m}{s}]$
				\item Momento angular: $\vec{L} = \vec{r} \times \vec{p}$, $\; [\vec{L}] = \frac{Kg m^{2}}{s}$
				\item Impulso: $\vec{J} = \Delta \vec{p} = \int_{t1}^{t2} \vec{F} dt$, $\; [\vec{p}] = [Kg] [\frac{m}{s}]$
				\item Centro de masa: $\vec{r}_{CM} = \frac{m_{1}}{m_{1} + m_{2}} \vec{r_{1}} + \frac{m_{2}}{m_{1} + m_{2}}
					\vec{r_{2}}$
				\item Conservación del momento lineal: $\vec{p_{i}} = \vec{p_{v}}$
				\item Conservación de la energía: $\frac{1}{2} m_{i} v_{i}^{2} = \frac{1}{2} m_{f} v_{f}^{2}$
				\item Energía cinética rotacional: $K = \frac{1}{2} I \omega^{2}$, donde $I = m r^{2}$ (Momento de Inercia)
				\item Momento de torsión (Torque): $\vec{\tau} = \vec{r} \times \vec{F} = \vec{r} \times (m\vec{a})$, $\; [\vec{\tau}] = J$
			\end{itemize}
		\end{multicols}

	\section*{\underline{Práctico 7:} Oscilaciones}
		\begin{itemize}
			\item
			\item Ley de Hooke: $\vec{F} = k \Delta \vec{x}$
			\item Trabajo realizado para estirar un resorte: $W_{e} = \frac{1}{2} k (\Delta l)^{2}$
		\end{itemize}

	\subsection{Partícula en movimiento armónico simple}
		\PN La posición de un objeto actuando sobre una fuerza descrita por la \textit{Ley de Hooke} esta dada por:
		\begin{equation}
		\end{equation}

		\PN donde A, $\omega$, son constantes y:
		\begin{itemize}
			\item Frecuencia angular: $\omega = \sqrt{\frac{k}{m}}$
			\item Período del movimiento: $T = \frac{2\pi}{\omega} = 2\pi \sqrt{\frac{m}{k}}$
			\item Frecuencia: $f = \frac{1}{T} = \frac{1}{2\pi} \sqrt{\frac{k}{m}}$
			\item Posición: $x(t) = A \cos (\omega t)$
			\item Velocidad: $v(t) = \frac{dx}{dt} = - \omega A \sin (\omega t)$
			\item Aceleración: $a(t) = \frac{d^{2}x}{dt^{2}} = - \omega^{2} A \cos (\omega t)$
		\end{itemize}

	\subsection{Energía del oscilador armónico simple}
		\PN \textbf{Energía cinética}
		\begin{equation}
			K = \frac{1}{2} m v^{2} = \frac{1}{2} m \omega^{2} A^{2} \sin^{2} (\omega t)
		\end{equation}

		\PN \textbf{Energía potencial}
		\begin{equation}
			U = \frac{1}{2} k x^{2} = \frac{1}{2} k A^{2} \cos^{2} (\omega t)
		\end{equation}

		\PN \textbf{Energía total}
		\PN Dado que K y U siempre son cantidades positivas o cero. Puesto que $\omega^{2} = \frac{k}{m}$, la energía mecánica
		total del oscilador armónico simple se expresa como:
		\begin{eqnarray*}
			E = \frac{1}{2} k A^{2}
		\end{eqnarray*}

	\subsection{Osciladores amortiguados}
		\PN Se puede escribir la segunda ley de Newton como:
		\begin{eqnarray*}
			\Sigma \ F_{x} &=& -kx - \underbrace{b \overbrace{v}^{\text{coeficiente de amortiguamiento}}}_{\text{fuerza retardadora}} = ma_{x} \\
			-kx - b \frac{dx}{dt} &=& m \frac{d^{2}x}{dt^{2}}
		\end{eqnarray*}

		\PN \textbf{Ecuación de la posición}
		\begin{equation}
			x(t) = A \mathrm{e}^{\frac{-b}{2m}t} \cos (\omega t)
		\end{equation}

		\PN donde la frecuencia angular de oscilación es:
		\begin{equation}
			\omega = \sqrt{\frac{k}{m} - \left(\frac{b}{2m}\right)^{2}}
		\end{equation}

		\PN Es conveniente expresar la frecuencia angular de un oscilador amortiguado en la forma:
		\begin{equation}
			\omega = \sqrt{\omega_{0} - \left(\frac{b}{2m}\right)^{2}}
		\end{equation}

		\PN donde $\omega_{0} = \sqrt{k/m}$ representa la frecuencia angular en ausencia de una fuerza retardadora (el
		oscilador no amortiguado) y se llama \textbf{frecuencia natural} del sistema.

		\pagebreak
		\PN \textbf{Tipos de amortiguamiento}
		\begin{itemize}
			\item Subamortiguado: $\omega_{0} > \frac{b}{2m}$
			\item Críticamente Amortiguado: $\omega_{0} = \frac{b}{2m}$
			\item Sobreamortiguado: $\omega_{0} < \frac{b}{2m}$
		\end{itemize}

	\subsection{Oscilaciones forzadas}
		\begin{itemize}
			\item Posición: $x(t) = A \cos (\omega t)$
			\item Velocidad: $v(t) = - \omega A \sin (\omega t)$
			\item lalala	: $a(t) = - \omega^{2} A \cos (\omega t)$
		\end{itemize}

	\section*{\underline{Práctico 8:} Calor}
		\begin{itemize}
			\item
		\end{itemize}

	\section*{\underline{Práctico 9:} Electroestática}
		\begin{itemize}
			\item
		\end{itemize}

	\section*{\underline{Práctico 10:} Ley de Gauss. Circuitos}
		\begin{itemize}
			\item
		\end{itemize}

	\section*{\underline{Práctico 11:} Magnetismo}
		TBD for final exam!!!
\end{document}
