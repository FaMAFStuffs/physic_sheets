\documentclass[12pt,a4paper]{article}
\usepackage[utf8]{inputenc}[spanish]
\usepackage{amsmath}
\usepackage{amsfonts}
\usepackage{amssymb}
\usepackage{lmodern}
\usepackage{amsmath}
\usepackage{amsthm}
\usepackage{enumerate}
\usepackage{graphicx}
\usepackage{mathtools}
\usepackage{stackrel}
\usepackage{multicol}
\usepackage[left=1cm,right=1cm,top=1cm,bottom=1cm]{geometry}
\newcounter{neq}
\providecommand{\abs}[1]{\lvert#1\rvert}
\newcommand{\SIGMA}{\Sigma^{\ast}}
\newcommand{\PN}{\par\noindent}

\title{Resumen Física}

\begin{document}
	\begin{center}
		\Huge \textbf{Resumen Física: Segundo Parcial} \\
		\large Agustín Curto
	\end{center}

	\section*{\underline{Práctico 1:} Cinemática}
		\begin{multicols}{2}
			\begin{itemize}
				\item Velocidad media: $\overline{v} = \frac{\Delta x}{\Delta t}$
				\item Aceleración media: $\overline{a} = \frac{\Delta v}{\Delta t}$
			\end{itemize}
		\end{multicols}

		\subsubsection*{Ecuaciones de movimiento:}
			\begin{multicols}{3}
				\begin{itemize}
					\item \underline{Posición:} $x(t) = x_{0} + v_{0} t + \frac{a}{2} t^{2}$
					\item \underline{Velocidad:} $v(t) = v_{0} + at$
					\item \underline{Aceleración:} $a = cte$
				\end{itemize}
			\end{multicols}

	\section*{\underline{Práctico 2:} Movimiento en el plano}
		\begin{itemize}
			\item Trayectoria: $y(x)$.
		\end{itemize}

		\subsubsection*{Ecuaciones de movimiento:}
			\begin{itemize}
				\item \underline{Posición:} $\vec{r}(t) = x(t) \hat{i} + y(t) \hat{j}$
				\item \underline{Velocidad:} $\vec{v}(t) = v_{x}(t) \hat{i} + v_{y}(t) \hat{j}$
				\item \underline{Aceleración:} $\vec{a}(t) = a_{x}(t) \hat{i} + a_{y}(t) \hat{j}$
			\end{itemize}

	\section*{\underline{Práctico 3:} Movimiento circular}
		\PN $v = \abs{\vec{v}}$
		\begin{multicols}{2}
			\begin{itemize}
				\item Aceleración: $\vec{a} = \vec{a_{c}} + \vec{a_{t}} \;$ donde:
					\begin{itemize}
						\item $\vec{a_{c}} = \frac{v^{2}}{r}$
						\item $\abs{\vec{a_{c}}} = r\gamma = r \frac{d\omega}{dt} = r \ddot{\theta}$
						\item $\vec{a_{t}} = \frac{d\vec{v_{t}}}{dt}$
						\item $\abs{\vec{a_{t}}} = r\omega^{2} = r \dot{\theta^{2}}$
					\end{itemize}
				\item Velocidad angular: $\omega = \frac{v}{r} \; [\frac{rad}{sec}]$
				\item Velocidad tangencial: $v_{t} = \omega r \; [\frac{mts}{sec}]$
				\item Período: $T = \frac{2\pi}{\omega}$
				\item Frecuencia: $f = \frac{1}{T}$
				\item Perímetro: $P = 2\pi r$
			\end{itemize}
		\end{multicols}

		\subsubsection*{Ecuaciones de movimiento en coordenadas polares:}
			\PN $\hat{r} = \neg\hat{n}$, $\hat{\theta} = \hat{t}$. $\dot{r} = \frac{dr}{dt}$, $\dot{\theta} =
				\frac{d\theta}{dt} = \omega$.
			\begin{itemize}
				\item \underline{Posición:} $\vec{r}(t) = r(t) \hat{r}$
				\item \underline{Velocidad:} $\vec{v}(t) = \dot{r}(t) \hat{r} + r(t) \dot{\theta}(t) \hat{\theta}$
				\item \underline{Aceleración:} $\vec{a}(t) = (\ddot{r}(t) - r(t) \dot{\theta}^{2}(t)) \hat{r} + (r(t)
					\ddot{\theta}(t) + 2 \dot{r}(t) \dot{\theta}(t)) \hat{\theta}$
			\end{itemize}

	\section*{\underline{Práctico 4:} Dinámica}
		\begin{multicols}{2}
			\begin{itemize}
				\item Leyes de Newton:
				\begin{enumerate}
					\item $\sum_{i} \vec{F_{i}} = 0 \Rightarrow \vec{a} = 0$ y $\vec{v} = cte$
					\item $\vec{F} = m \vec{a} \Rightarrow \sum \vec{F} = m \vec{a}$
				\end{enumerate}
				\item Fuerza gravitacional: $\vec{P} = m \vec{g}$
				\item Fuerza de rozamiento: $\abs{\vec{F_{R}}} = \mu \abs{\vec{N}}$
				\item Fuerza centrípeta: $\vec{F_{c}} = m \vec{a_{c}}$
				\item Resortes: $F = k_{e} \Delta x$
				\begin{itemize}
					\item En paralelo: $k_{e} = \sum_{i} k_{i}$
					\item En serie: $k_{e} = \frac{1}{\sum_{i} \frac{1}{k_{e}}}$
				\end{itemize}
			\end{itemize}
		\end{multicols}

	\section*{\underline{Práctico 5:} Trabajo y Energía}
		\begin{multicols}{2}
			\begin{itemize}
				\item Trabajo: $[W] = [\frac{N}{m}] = [J] \; (Joules)$
					\begin{itemize}
						\item $W = \int_{P_{i}}^{P_{f}} \vec{F} d\vec{s}$
						\item $W = F \Delta x$
					\end{itemize}
				\item Energía Cinética: $K = \frac{1}{2} m v^{2}$, $\; [K] = [J]$
				\item Energía Potencial: $U = mgh$, $\; [U] = [J]$
				\item Teorema de Energía-Trabajo:
					\[
						F \Delta x = \frac{1}{2} m v_{f}^{2} - \frac{1}{2} m v_{i}^{2}
					\]
				\item Conservación de la Energía:
					\begin{eqnarray*}
						E_{inicial} &=& E_{final} \\
						K_{i} + U_{i} &=& K_{f} + U_{f}
					\end{eqnarray*}

				\item Potencia: $[P] = [\frac{J}{s}] = [W] \; (Watt)$
					\begin{itemize}
						\item Potencia media: $\overline{P} = \frac{W}{\Delta t}$
						\item Potencia instantanea: $\vec{P} = \frac{dT}{dt} = \vec{F} \cdot \vec{v}$
					\end{itemize}
			\end{itemize}
		\end{multicols}

	\section*{\underline{Práctico 6:} Momento Lineal, Angular y de Torsión}
		\begin{multicols}{2}
			\begin{itemize}
				\item Momento lineal: $\vec{p} = m \vec{v} \qquad [\vec{p}] = [\frac{kg \; m}{s}]$
				\item Momento angular: $\vec{L} = \vec{r} \times \vec{p} \qquad [\vec{L}] = [\frac{Kg \; m^{2}}{s}]$
				\item Impulso: $\vec{J} = \Delta \vec{p} = \int_{t1}^{t2} \vec{F} dt \qquad [\vec{p}] = [\frac{kg \; m}{s}]$
				\item Centro de masa: $\vec{r}_{CM} = \frac{m_{1}}{m_{1} + m_{2}} \vec{r_{1}} + \frac{m_{2}}{m_{1} + m_{2}}
					\vec{r_{2}}$
				\item Choque plástico:
					\begin{itemize}
						\item Se conserva el momento
						\item No se conserva la energía
					\end{itemize}
				\item Choque elástico:
					\begin{itemize}
						\item Se conserva el momento
						\item Se conserva la energía
					\end{itemize}
				\item Conservación del momento lineal: $\vec{p_{i}} = \vec{p_{v}}$
				\item Conservación de la energía: $\frac{1}{2} m_{i} v_{i}^{2} = \frac{1}{2} m_{f} v_{f}^{2}$
				\item Energía cinética rotacional: $K = \frac{1}{2} I \omega^{2}$
					\PN $I = m r^{2}$ (Momento de Inercia) $[I] = [Kg \; m^{2}]$
				\item Momento de torsión (Torque):
					\[
						\vec{\tau} = \vec{r} \times \vec{F} = \vec{r} \times (m\vec{a}) \qquad [\vec{\tau}] = J
					\]
			\end{itemize}
		\end{multicols}

	\section*{\underline{Práctico 7:} Oscilaciones}
		\begin{itemize}
			\item Ley de Hooke: $\vec{F} = k \Delta \vec{x}$
			\item Trabajo realizado para estirar un resorte: $W_{e} = \frac{1}{2} k (\Delta l)^{2}$
			\item Oscilador Armónico Simple:
				\begin{multicols}{2}
					\begin{itemize}
						\item Posición: $x(t) = A \cos (\omega t + \phi)$
						\item Velocidad: $v(t) = - \omega A \sin (\omega t + \phi)$
						\item Aceleración: $a(t) = - \omega^{2} A \cos (\omega t + \phi)$
						\item Frecuencia angular: $\omega = \sqrt{\frac{k}{m}}$
						\item Período del movimiento: $T = \frac{2\pi}{\omega} = 2\pi \sqrt{\frac{m}{k}}$
						\item Frecuencia: $f = \frac{1}{T} = \frac{1}{2\pi} \sqrt{\frac{k}{m}}$
						\item Valores máximos:
							\begin{itemize}
								\item $x_{max} = A$
								\item $v_{max} = \omega A$
								\item $a_{max} = \omega^{2}A$
							\end{itemize}
						\item Energía: $E = \frac{1}{2} k A^{2}$
						\item Velocidad en función de la posición:
							\[
								v(x) = \pm \omega \sqrt{A^{2} - x^{2}}
							\]
					\end{itemize}
				\end{multicols}
			\item Oscilador Amortiguado
				\begin{multicols}{2}
					\begin{itemize}
						\item $x(t) = A \mathrm{e}^{\frac{-b}{2m}t} \cos (\omega t)$
						\item $\omega = \sqrt{\frac{k}{m} - \left(\frac{b}{2m}\right)^{2}} = \sqrt{\omega_{0} - \alpha^{2}}$
						\item Tipos de amortiguamiento:
						\begin{itemize}
							\item Subamortiguado: $\omega_{0} > \alpha$
							\item Sobreamortiguado: $\omega_{0} < \alpha$
							\item Críticamente Amortiguado: $\omega_{0} = \alpha$
						\end{itemize}
					\end{itemize}
				\end{multicols}
			\item Oscilaciones forzadas
				\begin{multicols}{3}
					\begin{itemize}
						\item $x(t) = A \cos (\omega t + \phi)$
						\item $v(t) = - \omega A \sin (\omega t + \phi)$
						\item $a(t) = - \omega^{2} A \cos (\omega t + \phi)$
					\end{itemize}
				\end{multicols}
		\end{itemize}

	\section*{\underline{Práctico 8:} Calor}
		\begin{itemize}
			\item Celsius, Farenheit y Kelvin:
				\begin{multicols}{2}
					\begin{itemize}
						\item $T_{C} = T_{K} - 273.15$
						\item $T_{F} = \frac{9}{5} T_{C} + 32$
					\end{itemize}
				\end{multicols}
			\item Expansión térmica. $\alpha$, coeficiente de expanción.
				\begin{multicols}{3}
					\begin{itemize}
						\item 1D: $ \Delta L = \alpha L_{i} \Delta T$
						\item 2D: $ \Delta A = 2\alpha A_{i} \Delta T$
						\item 3D: $ \Delta V = 3\alpha V_{i} \Delta T$
					\end{itemize}
				\end{multicols}
			\item Calor: $Q = C \Delta T = cm \Delta T \qquad [Q] = J$.
			\item Calorias-Joules: $1 [cal] = 4.186 [J]$
			\item Calor latente: $Q = L \Delta m$
				\begin{itemize}
					\item Fusión: $Q = L_{F} \Delta m$, $\qquad$ para el agua: $L_{F} = 80 [\frac{cal}{gr}]$
					\item Vaporización: $Q = L_{V} \Delta m$, $\qquad$ para el agua: $L_{V} = 540 [\frac{cal}{gr}]$
				\end{itemize}
			\item Transferencia de calor:
		\end{itemize}

	\section*{\underline{Práctico 9:} Electroestática}
		\begin{itemize}
			\item
		\end{itemize}
\end{document}
